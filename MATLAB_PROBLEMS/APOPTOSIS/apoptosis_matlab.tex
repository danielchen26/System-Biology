% important: seems to need graphicx, does not woprk with graphics; maybe
%         because using png (and old notes have jpg only, so no problem there?)
\def\FIGDIR{figdir_ingalls}
\documentclass[12pt]{article}

\newcommand{\ans}[1]{}
%\renewcommand{\ans}[1]{
%
%{\bf ANSWER:} #1\medskip}

\usepackage{color}  %for changes
\newcommand{\edo}{\printindex\end{document}}
\usepackage{times}
\usepackage{epsf,latexsym,amssymb}

\topmargin -1in
\textheight 9.7in
\oddsidemargin -0.25in
\evensidemargin -0.25in
\textwidth 7in
\parskip=5pt plus 1pt minus 1pt
\parindent0pt


\newcommand{\comment}[1]{}

\newcommand{\pic}[2]{\includegraphics[scale=#1]{\FIGDIR/#2}}
\newcommand{\picc}[2]{\centerline{\pic{#1}{#2}}}

\newcommand{\beq}{\begin{eqnarray}}
\newcommand{\eeq}{\end{eqnarray}}
\newcommand{\beqn}{\begin{eqnarray*}}
\newcommand{\eeqn}{\end{eqnarray*}}
\newcommand{\bi}{\begin{itemize}}
\newcommand{\ei}{\end{itemize}}
\newcommand{\ben}{\begin{enumerate}}
\newcommand{\een}{\end{enumerate}}
\newcommand{\be}[1]{\begin{equation}\label{#1}}
\newcommand{\ee}{\end{equation}}
\newcommand{\abs}[1]{\left\vert{#1}\right\vert}

\newcommand{\arrowchem}[1]{{\stackrel{\displaystyle#1}{\longrightarrow}}}
\newcommand{\arrowschem}[2]{\raisebox{-2ex}%
	{$\stackrel{\stackrel{\displaystyle#1}{\longrightarrow}}%
	{\stackrel{\longleftarrow}{#2}}$}}
\newcommand{\minp}[2]{\begin{minipage}{#1\textwidth}#2\end{minipage}}
\newcommand{\mypmatrix}[1]{\left(\begin{array}{cccccccccccc}#1\end{array}\right)}




% new commands specific to this file:
%\usepackage{graphics}
\usepackage{graphicx}
\newcommand{\tim}{} % could be (t)
%\newcommand{\CER}{\mbox{C}_{\mbox{\sc er}}}
%\newcommand{\RICC}{\mbox{RIC}^+\mbox{C}^-}
%\newcommand{\RIC}{\mbox{RIC}^+}
%\newcommand{\RI}{\mbox{RI}}
%\newcommand{\CC}{\mbox{C}}
%\newcommand{\RR}{\mbox{R}}

\begin{document}

\subsection*{Apoptosis}

The process of programmed cell death --cellular suicide-- is called apoptosis
(from the Greek for ``a falling off''). Apoptosis is a necessary part of the
development of many multicellular organisms.
Some cells play only a transient role during development; when they are no
longer needed, they receive signals that induce apoptosis. (A commonly cited
example is the tail of a tadpole, which is not needed by the adult frog.)
Compared with death caused by stress or injury, apoptosis is a tidy process;
rather than spill their contents into the environment, apoptotic cells quietly
implode, thus ensuring there are no detrimental effects on the surrounding
tissue. 

Apoptosis is invoked by caspase proteins, which are always present in the
cell, but lie dormant until activated. The family of caspase proteins is split
into two categories:
%\bi
%\item

$\bullet$
Initiator caspases respond to apoptosis-inducing stimuli. They can be
triggered externally (via transmembrane receptors) or internally, by stress
signals from the mitochondria (e.g., starvation signals).

$\bullet$
Executioner caspases are activated by initiator caspases. They carry out the
task of cellular destruction by cleaving a number of key proteins and
activating DNases that degrade the cell's DNA.

We will consider a model published in:

\emph{Eissing,T., Conzelmann,H., Gilles, E.D., Allg\"ower, F., Bullinger,E., and
Scheurich,P., Bistability analyses of a caspase activation model for
receptor-induced apoptosis. Journal of Biological Chemistry, 279(2004),
36892-36897.}

The model focuses on caspase-8, an initiator, and caspase-3, an
executioner. Caspase-8 is triggered by external stimuli. When active, it
activates caspase-3. Caspase proteins are activated by the removal of a
masking domain, revealing a catalytic site. This cleavage is irreversible; the
protein can only be inactivated by degradation. Consequently, to describe
steady-state behavior, the model includes both production and degradation
processes for each protein.

To guarantee that the decision to undergo apoptosis is irreversible, a
feedback mechanism is in place: active caspase-3 activates caspase-8. This
positive feedback ensures that caspase activity is self-perpetuating. The
feedback scheme in the figure shows the core reaction
network for the Eissing model. In addition to the caspases, the model
incorporates two proteins, IAP and BAR, which inhibit apoptosis by binding
active caspases, forming inert complexes.

\picc{0.3}{fig615_ingalls_apoptosis.png}

\emph{Eissing apoptosis model (shown not as a chemical reaction network, but 
using a different graphical representation). An extracellular signal triggers
  activation of 
  caspase-8 (C8*) from its inactive form (C8). Once active, caspase-8
  activates caspase-3 (C3 to C3*). Active caspase-3 activates caspase-8,
  forming a positive feedback loop. Because activation of caspases is
  irreversible, protein degradation and production are included in the
  model. Two apoptotic inhibitors are included: BAR and IAP. These proteins
  bind active caspase-8 and -3, respectively, thus inhibiting the progression
  to apoptosis. (Dots indicate degraded proteins.)}

%Adapted from Figure 1 of
% Eissing et al., 2004.}

(Dashed lines represent enzymatic reactions like $C3^*+C8 -> C3^* + C8^*$ which
do not consume the enzyme.  The authors implicitly assume these
occur fast enough that we ignore intermediate complexes.)

Treating all enzyme-catalyzed reactions as first-order, we can write the model
as
\beqn
d/dt[\mbox{C8}]\tim &=& 
k_{1}- 
k_{2}[\mbox{C8}]\tim - 
k_{3}([\mbox{C3} ^* ]\tim +
[\mbox{Input}]\tim )!.[\mbox{C8}]\tim 
\\
d/dt[\mbox{C8} ^* ]\tim &=&
k_{3}([\mbox{C3} ^* ]\tim +
[\mbox{Input}]\tim )!.[\mbox{C8}]\tim - 
k_{4}[\mbox{C8} ^* ]\tim - 
k_{5}[\mbox{C8} ^* ]\tim !.[\mbox{BAR}]\tim +
k_{6}[\mbox{C8} ^* \mbox{BAR}]\tim  
\\
d/dt[\mbox{C3}]\tim &=& 
k_{7}- 
k_{8}[\mbox{C3}]\tim - 
k_{9}[\mbox{C8} ^* ]\tim !.[\mbox{C3}]\tim 
\\
d/dt[\mbox{C3} ^* ]\tim &=&
k_{9}[\mbox{C8} ^* ]\tim !.[\mbox{C3}]\tim - 
k_{10}[\mbox{C3} ^* ]\tim - 
k_{11}[\mbox{C3} ^* ]\tim !.[\mbox{IAP}]\tim +
k_{12}[\mbox{C3} ^* IAP]\tim 
\\
d/dt[\mbox{BAR}]\tim &=& k_{13}-
k_{5}[\mbox{C8} ^* ]\tim !.[\mbox{BAR}]\tim +
k_{6}[\mbox{C8} ^* \mbox{BAR}]\tim -
k_{14}[\mbox{BAR}]\tim 
\\
d/dt[\mbox{IAP}]\tim &=& k_{15}-
k_{11}[\mbox{C3} ^* ]\tim !.[\mbox{IAP}]\tim +
k_{12}[\mbox{C3} ^* \mbox{IAP}]\tim - 
(k_{16}+
k_{17}[\mbox{C3} ^* ]\tim )!.[\mbox{IAP}]\tim 
\\
d/dt[\mbox{C8} ^* \mbox{BAR}]\tim  &=& 
k_{5}[\mbox{C8} ^* ]\tim !.[\mbox{BAR}]\tim - 
k_{6}[\mbox{C8} ^* \mbox{BAR}]\tim- 
k_{18}[\mbox{C8} ^* \mbox{BAR}]\tim 
\\
d/dt[\mbox{C3} ^* \mbox{IAP}]\tim &=&
k_{11}[\mbox{C3} ^* ]\tim !.[\mbox{IAP}]\tim - 
k_{12}[\mbox{C3} ^* \mbox{IAP}]\tim - 
k_{19}[\mbox{C3} ^* \mbox{IAP}]\tim 
\eeqn
This system is bistable. At low levels of activated caspase, the system is at
rest in a ``life'' state.  Once caspase activity rises above a threshold, the
positive feedback commits the system to reaching a steady state with high
levels of active caspase -- a ``death'' state. (Of course, the death state is
transient - the cell is being dismantled. We are justified in calling it a
steady state on the timescale of the signaling pathway.)

The simulation of the model shows the response of the system
to an input signal.  The system begins at rest with zero input and low caspase
activity. At time t = 100 minutes an input is introduced, causing a slow
increase in the activity level of caspase-8. This slow activation leads to a
rapid rise in caspase activity at about t = 300 minutes. The system then
settles into the ``death'' state with high caspase activity. When the input
signal is removed (at time t = 1200 minutes), this self-perpetuating state
persists, confirming that the system is bistable. Because complete removal of
the input signal does not cause a return to the initial state, this
life-to-death transition is irreversible.

\picc{0.4}{fig616_ingalls_apoptosis.png}

\emph{Behavior of the Eissing apoptotic pathway model. This simulation
begins in the low caspase activity ``life'' state.  At time t =100 minutes, an
input signal (Input=200) is introduced, causing an increase in caspase-8
activity. This triggers a positive feedback loop that results in a rapid
increase in activated caspase-8 and caspase-3 abundance at about t = 300. The
system then settles to the high caspase-activity ``death'' state. The input
stimulus is removed at time t =1200 minutes, but there is no effect: the
apoptotic switch is irreversible.
Parameter values:
(in mpc/min) $k_{1} = 507, k_{7} = 81.9, k_{13} = 40, k_{15}= 464$; 
(in min-1) $k_{2} =3.9 !p 10^{-3} , k_{4} =5.8 !p 10^{-3} , k_{6} =0.21, k_{8}
=3.9 !p 10^{-3} , k_{10} =5.8 !p 10^{-3} , k_{12} =0.21, k_{14} =1!p 10^{-3},
k_{16} =1.16!p 10^{-2} , k_{18} =1.16!p 10^{-2} , k_{19} =1.73!p 10^{-2}$;
in 1/(mpc.min)
$k_{3} =1!p 10^{-5} , k_{5} =5!p 10^{-4} , k_{9} =5.8!p 10^{-6}, 
k_{11} =5!p10^{-4} , k_{17} =3!p 10^{-4}$ (mpc =molecules per cell).
}


\newpage
\subsection*{Problems}

\ben
\item
Write the chemical reaction network (e.g., $BAR+C8^* \leftrightarrow C8^*BAR$,
etc.) that gives rise to the equations that were given.  
\item
For this CRN, write the vector of species, the stoichiometric matrix, and the
vector of reaction rates, and verify that, multiplying out, one gets the
equations that were given.
\item
%6.8.10 
(Duration of triggering signal.)
The simulation shown in the figure is done using
the MATLAB program \emph{apoptosis\_eissing\_model.m}.
The input signal was maintained for 1100 minutes (until after the system had
settled to the caspase-active ``death'' state.) 
Re-run this simulation many times, with different durations,
until you determine what is the shortest input pulse can
trigger the irreversible life-to-death transition. Use the same
input size as in the figure (Input=200).  Your answer should be an integer,
and you should print out a plot showing the behavior with this duration and
a plot showing the behavior with 1 minute less.
(For example, if you find that 323 minutes is the smallest duration that
works, show the plots for 322 and for 323.)

Be smart about how you do the search - a bisection method worked quite well
for me.

\ans{I get roughly 125 minutes as the shortest that works.  At 125, I get a
graph very much as the one in the figure, but at 124, I get a flat line at
zero, basically.}

\item
%6.4.1 
Explain what one means by this statement:
``In the model, active caspase-3 promotes degradation of IAP.  This interaction
can be seen as an additional positive feedback of $C3^*$ on itself.''
Your answer should be a couple of sentences explaining why the degradation of
IAP implies that $C3^*$ gets to stay around longer than if IAP was not degraded.

\ans{IAP binds active caspase-3, removing it from the pathway.  By enhancing
degradation of IAP, active caspase-3 increases its own concentration, because
there is less IAP to bind it. }

Therefore, this degradation helps self-sustained caspase activity, hence also
helping the positive feedback that makes caspase activation irreversible.

\item
Which $k_i$ should be set to zero to get rid of the effect of
caspase-3 promoting degradation of IAP?

\ans{$k_{17}$}

\item
Simulate (with an input of length 1100 and magnitude 200) what happens when the
$k_i$ from the previous problem is set to zero.  Plot the result and print it.

\ans{

\picc{0.4}{apoptosis_no_degradation_plot_over_2500sec.png}
}
\een
%
%%6.8.11
%Additional problem, on model reduction, not assigned:
%In a 2007 paper, Steffen Waldherr and colleagues presented a reduced version
%of the apoptosis model shown above.  The authors determined that bistability
%is retained when a quasi-steady state assumption is applied to four of the
%state variables.  Verify their finding, as follows. Apply a quasi-steady state
%assumption to the species [C8], [C3], [IAP] and [BAR] and demonstrate
%bistability in the reduced model.  (This model reduction was not motivated by
%a separation of time-scales. Instead, Waldherr and colleagues determined the
%significance of each state variable for bistability, and eliminated those that
%were not needed.)

\end{document}






