% important: seems to need graphicx, does not woprk with graphics; maybe
%         because using png (and old notes have jpg only, so no problem there?)
\def\FIGDIR{figdir_ingalls}
\documentclass[12pt]{article}

%\newcommand{\ans}[1]{}
\newcommand{\ans}[1]{

{\bf ANSWER:} #1\medskip}

\usepackage{color}  %for changes
\newcommand{\edo}{\printindex\end{document}}
\usepackage{times}
\usepackage{epsf,latexsym,amssymb}

\topmargin -0.8in
\textheight 9.4in
\oddsidemargin 0.0in
\evensidemargin 0.0in
\textwidth 6.7in
\parskip=5pt plus 1pt minus 1pt
\parindent0pt


\newcommand{\comment}[1]{}

\newcommand{\pic}[2]{\includegraphics[scale=#1]{\FIGDIR/#2}}
\newcommand{\picc}[2]{\centerline{\pic{#1}{#2}}}

\newcommand{\beq}{\begin{eqnarray}}
\newcommand{\eeq}{\end{eqnarray}}
\newcommand{\beqn}{\begin{eqnarray*}}
\newcommand{\eeqn}{\end{eqnarray*}}
\newcommand{\bi}{\begin{itemize}}
\newcommand{\ei}{\end{itemize}}
\newcommand{\ben}{\begin{enumerate}}
\newcommand{\een}{\end{enumerate}}
\newcommand{\be}[1]{\begin{equation}\label{#1}}
\newcommand{\ee}{\end{equation}}
\newcommand{\abs}[1]{\left\vert{#1}\right\vert}

\newcommand{\arrowchem}[1]{{\stackrel{\displaystyle#1}{\longrightarrow}}}
\newcommand{\arrowschem}[2]{\raisebox{-2ex}%
	{$\stackrel{\stackrel{\displaystyle#1}{\longrightarrow}}%
	{\stackrel{\longleftarrow}{#2}}$}}
\newcommand{\minp}[2]{\begin{minipage}{#1\textwidth}#2\end{minipage}}
\newcommand{\mypmatrix}[1]{\left(\begin{array}{cccccccccccc}#1\end{array}\right)}




% new commands specific to this file:
%\usepackage{graphics}
\usepackage{graphicx}
\newcommand{\tim}{} % could be (t)
\newcommand{\CER}{\mbox{C}_{\mbox{\sc er}}}
\newcommand{\RICC}{\mbox{RIC}^+\mbox{C}^-}
\newcommand{\RIC}{\mbox{RIC}^+}
\newcommand{\RI}{\mbox{RI}}
\newcommand{\CC}{\mbox{C}}
\newcommand{\RR}{\mbox{R}}

\begin{document}



\subsection*{MATLAB project: calcium oscillations}

\ans{VERSION WITH ANSWERS}

Many types of animal cells use calcium ions, Ca$2^+$, as part of signal
transduction cascades. Calcium is used to trigger, for example, the initiation
of embryonic development in fertilized egg cells, the contraction of muscle
cells, and the secretion of neurotransmitters from neurons.

Calcium signals are sent by rapid spikes in cytosolic Ca$2^+$ concentration.
Cells that employ these signals normally have low levels of cytosolic calcium
(about 10-100 nM). These low levels are maintained by ATP-dependent pumps that
export cytosolic Ca$2^+$ both out of the cell and into the endoplasmic
reticulum (ER). The concentration of Ca$2^+$ in the ER can reach as high as 1
mM (106 nM). Signaling pathways open calcium channels in the ER membrane,
leading to rapid (diffusion-driven) surges in cytosolic calcium levels.

However, because calcium is involved in many cellular processes, persistent
high concentrations can be detrimental. (For example, failure to remove
calcium from muscle cells keeps them in a state of constant tension.  This is
what causes rigor mortis.) Some cells that use calcium as an intracellular
signaling molecule avoid persistently high Ca$2^+$ concentrations by generating
oscillations in calcium levels. The frequency of the oscillations is dependent
on the intensity of the signal, while the amplitude is roughly constant. The
downstream cellular response is dependent on the oscillation frequency.

We will consider an instance of this frequency-encoding mechanism in mammalian
liver cells.  These cells respond to certain hormones with the activation of
G-protein-coupled receptors (Section 6.1.2). The G-protein triggers a
signaling pathway that results in production of inositol 1,4,5-triphosphate
(IP3). These IP3 molecules bind a receptor that is complexed with a calcium
channel in the membrane of the ER.

The IP3 binding event exposes two receptor sites at which Ca$2^+$ ions can bind
(Figure 6.17).  These two sites have different affinities for Ca$2^+$.  At low
concentration only one site is occupied, while at higher concentrations both
sites are bound. The calcium binding events have opposing effects on the
receptor-channel complex.  Binding of the first calcium ion causes the channel
to open, allowing Ca$2^+$ to flow into the cytosol.  Binding of the second ion
causes the channel to close. This interplay of positive and negative feedback
generates oscillations in the cytosolic Ca$2^+$ concentration, as follows. When
cytosolic calcium levels are low, the channels are primarily in the open
state, and so Ca$2^+$ rushes into the cytosol from the ER. When high calcium
levels are reached, the channels begin to shut. Once most of the channels are
closed, the continual action of the Ca$2^+$ pumps eventually causes a return to
low cytosolic [Ca$2^+$], from which the cycle repeats.

In 1993 Hans Othmer and Yuanhua Tang developed a model of this pathway that
focuses on the behaviour of the channel, described in:

\emph{H.G. Othmer, Signal transduction and second messenger systems. In Case
  Studies in Mathematical Modelling: Ecology, Physiology, and Cell Biology,
  H.G. Othmer, F.R. Adler, M.A. Lewis and J. C. Dallon, eds, 99-126, 1996}

Taking I =[IP3] as the system input, the receptor binding events are described
by:
\[
\mbox{I} + \mbox{R} \;\arrowschem{k_1}{k_{-1}}\; \RI
%\]
%\[
\quad\quad\quad\quad
\RI + \mbox{C} \;\arrowschem{k_2}{k_{-2}}\; \RIC 
%\]
\quad\quad\quad\quad
%\[
\RIC + \mbox{C}  \;\arrowschem{k_3}{k_{-3}}\; \RICC
\]
where R is the receptor-channel complex, C is cytosolic calcium, \RI\ is the
IP3-bound receptor- channel complex, $\RIC$ is the open (one Ca$2^+$-bound)
channel, and $\RICC$ is the closed (two Ca$2^+$-bound) channel.

The rate of diffusion of calcium into the cytosol depends on the concentration
of calcium in the ER (denoted [$\CER$], and held fixed) and the abundance of open
channels.  The rate of diffusion is proportional to the difference in
concentration between the two compartments. This transport rate is modeled as 
\[
\mbox{rate of Ca$2^+$ diffusion into the cytosol
   $=v_r(\gamma _0+\gamma _1[\RIC])([\CER ]- [\CC])$}
\]
where $v_r$ is the ratio of the ER and cytosolic volumes, and $\gamma _0$
characterizes a channel-independent ``leak.'' 

\picc{0.3}{fig617_ingalls_calcium.png}

\emph{Calcium-induced calcium release. A G-protein pathway (not shown)
  responds to a hormone signal by inducing production of IP3, which activates
  calcium channels in the ER membrane. These channels bind Ca$2^+$ ions at two
  sites. The first binding event causes the channel to open; the second causes
  it to close. Calcium pumps continually pump Ca$2^+$ ions from the cytosol to
  the ER.}


Calcium is continually pumped from the cytosol to the ER.  Presuming strong
cooperativity of calcium uptake, the pumping rate is modeled as
\[
\mbox{rate of Ca$2^+$ pumping out of the cytosol 
$=\frac{p_1[\CC]^4}{p_2^4+[\CC]^4 }$}
\]
for parameters $p_1$ and $p_2$. 
The complete model is then: 
%vr*(gamma0+gamma1*S(4))*(Cs-S(1)) - (p1*S(1)^4)/(p2^4+S(1)^4)
%-k1*I*S(2)+km1*S(3)
%-(km1+k2*S(1))*S(3)+ k1*I*S(2) + km2*S(4)
%-(km2+k3*S(1))*S(4) + k2*S(1)*S(3) + km3*S(5)
%k3*S(1)*S(4) - km3*S(5)
\beqn
d/dt[\RR]\tim &=& -k_{1}[I]\tim \cdot [\RR]\tim +k_{-1}[\RI]\tim \\
d/dt[\RI]\tim &=& -(k_{-1}+ 
k_2[\CC]\tim )\cdot [\RI]\tim +
k_{1}[I]\tim \cdot [\RR]\tim +
k_{-2}[\RIC]\tim \\
d/dt[\RIC]\tim &=& -(k_{-2}+
k_3[\CC]\tim )\cdot [\RIC]\tim +
k_2[\CC]\tim \cdot [\RI]\tim +
k_{-3}[\RICC]\tim \\
d/dt[\RICC]\tim &=& k_3[\CC]\tim \cdot [\RIC]\tim - 
k_{-3}[\RICC]\tim \\
d/dt[\CC]\tim &=& v_r(\gamma _0+\gamma _1[\RIC])([\CER ]- [\CC])
- \frac{p_1[\CC]^4}{p_2^4+[\CC]^4 }
\eeqn

The following simulation in illustrates the system's oscillatory
behavior.  When the calcium level is low, the concentration of open channels
increases, followed by a rapid increase in [Ca$2^+$].
Once the calcium concentration rises, the channels close, and the calcium
level falls, setting up a new cycle. 


\picc{0.6}{calcium_main_illustration.png}
%{fig618_ingalls_calcium.png}

\emph{Calcium oscillations. This simulation shows the oscillatory
behaviour of the system. When calcium levels are low, channels open, letting
more Ca$2^+$ into the cytosol. As calcium levels rise, channels begin to close,
leading to a drop in cytosolic Ca$2^+$ levels. The IP3 concentration is fixed at
2 $\mu M$. B. In this simulation the input level of IP3 changes, demonstrating the
frequency-encoding ability of the system. As the IP3 level(in $\mu M$) increases,
the frequency of the oscillations increases considerably, while the amplitude
is roughly constant.
Parameter values used : (in $1/\mu M$ 1/s) $k_{1} = 12, k_2 = 15, k_3 =1.8$; (in
1/s) $k_{-1} = 8, k_{-2} =1.65, k_{-3} =0.21, \gamma _0 =0.1, \gamma _1 =20.5; [\CER ]=8.37
\mu M$, $p_1 =8.5$ $\mu M$ 1/s, $p_2 =0.065$ $\mu M$, $v_r =0.185$.}





%Panel B demonstrates the system's
%frequency-encoding ability. As the input is increased (in steps), the
%frequency of oscillations increases while the amplitude changes very little.

\newpage
\subsection*{Problems}
\ben
\item
For this problem, you will use the MATLAB program ``calcium\_cytosolic''
which was provided by the instructor. 

You have to place the file calcium\_cytosolic.m in a folder that MATLAB will
find.  To run the program, just type ``calcium\_cytosolic'' in MATLAB.

Please print and hand-in the corresponding plots.  It is OK to give a
very approximate answer to the questions about frequency.
\ben
\item
Look at the program listing.
(Open the file calcium\_cytosolic.m using MATLAB or your favorite text editor.)
What are the initial concentrations for the variables?  You don't need to
``know MATLAB'' to read the file and figure this out!
\ans{$R(0)=1$ (the total amount of receptors), and the rest all zero.}
\item
Run the program, which has an IP3 concentration set to $I=1$ at time $t=20$.
You will observe that, after stimulation, the system settles into an
oscillation in cytosolic calcium with a frequency of roughly 0.14 1/s.  This
is estimated as follows: on an interval of length 100 seconds, from time
$t=25$ to $t=125$ (more or less!{}) we have about 14 complete oscillations, and
14/100 = 0.14.
\item
Modify the program to set $I=0.7$.  (Open the file calcium\_cytosolic.m using
MATLAB or your favorite text editor, and change the line where $I$ was set to
1.)  
Now run the program, and answer: what frequency of oscillations results now?
\ans{0.11}
\item
Repeat with $I=2$.
\ans{0.19}
\item
Repeat with $I=10$.
\ans{0.24}
\item
Speculate on  what happens if the input $I$ is set really large.  Does the
frequency tend to infinity?
\ans{The frequency does not go to infinity.  In fact, biologically, because
IP3 has to bind to receptors, and there is a finite number of receptors, at
some point more IP3 will not make any difference.  The maximum frequency seems
to be roughly 0.25}
%\item
%Repeat with $I=0.4$.  What do you see now?
%\ans{There are no oscillations.}
\item
Repeat with $I=0.5$.  What do you see now?
\ans{There is a peak of activity but there are no oscillations.}
\item
Suppose now that (perhaps because of a mutation), calcium cannot
bind anymore to the open channels, so as to close them.
Answer: which parameter $k_i$ has to be changed to zero in order to model this
new system?
\ans{Set $k_3=0$}
\item
Now set the parameter $k_i$ from the previous part to zero, and run the
program again (with $I=1$).  {\bf Make sure} to change the ``2.5'' in the line:
\begin{verbatim}
axis([0 120 0 2.5])
\end{verbatim}
to something large, let us say 10, before running the program.  (This sets the
$y$ axis to be between 0 and 10.)  
Interpret in words what you see.
\ans{The concentration of cytosolic calcium approaches a high limit.  The
value is higher than earlier, because the channels remain open, so the calcium
does not re-enter the ER.}
\een
% next from Ingalls' book:
\item
%6.8.12 
Calcium-induced calcium release: frequency encoding. 
%Consider the model of calcium oscillations presented earlier.  
Run simulations to explore the dependence of frequency and amplitude on the
strength of the input.  Prepare plots showing frequency and amplitude as a
function of model input I over the range 1-10 $\mu M$. Does the amplitude vary
significantly over this range? What about for higher input levels?

\item
%6.8.13 
Calcium-induced calcium release: parameter balance.
%Consider the model of calcium oscillations presented earlier.  
The model's oscillatory behaviour depends on a balance between the positive
and negative effects of calcium binding.  Oscillations are easily lost if
parameter values vary. Explore this sensitivity by choosing one of the model
parameters and determining the range of values over which the system exhibits
oscillations(with I =1 $\mu M$). Provide an intuitive explanation for why
oscillations are lost outside the range you have identified.

\item
%6.8.14 
Calcium-induced calcium release: frequency decoding. 
One mechanism by which cells can ``de-code'' the information encoded in
calcium oscillations is by the activity of Ca$2^+$/calmodulin-dependent
protein kinases (CaM-kinases). Calmodulin is a protein that mediates a number
of calcium-dependent cellular processes. It has four high-affinity Ca$2^+$
binding sites, and is activated when saturated by Ca$2^+$ . CaM-kinases are
activated (by autophosphorylation) upon binding to active calmodulin. The
CaM-kinaseactivityis ``turned off'' byphosphatases.  Extend the model to
include CaM-kinase activity, and verify that, for persistent oscillations, the
higher the frequency ofCa$2^+$ oscillations, the higher the average CaM-kinase
activity level.  To keep the model simple, suppose that four calcium ions bind
calmodulin simultaneously (i.e. with high cooperativity), and that CaM-kinase
autophosphorylation occurs immediately upon calmodulin binding. (Hint: the
frequency-dependent effect is strongest when the time-scales of deactivation
of calmodulin and CaM-kinase are slow, so that high-frequency inputs cause
near-constant activity levels.)
\een

\end{document}
